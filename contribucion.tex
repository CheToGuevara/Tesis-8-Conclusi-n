\section{Validación de hipótesis}
\label{conclu:hipotesis}

El algoritmo propuesto ha cumplido con los objetivos que se habían enunciado en la sección \ref{intro:objetivos}. Se ha diseñado un cauce de animación esqueletal que permite transformar interactivamente los modelos anatómicos de pacientes virtuales, los cuales originalmente se encuentran en una postura diferente a la posición necesaria para el entrenamiento de un determinado procedimiento médico. A su vez, el cauce permite deformar la anatomía aunque los modelos estén incompletos o falten sus descripciones mecánicas. Aunque no se ha podido realizar una evaluación formal de la herramienta \ac{ITGVPH} en el proyecto \ac{RASimAs}, 
\todo{explica que contaba con el visto bueno del los medicos implicados en tel proyecoto}
%en el segundo caso de uso si ha sido posible. 
\todo{Reescribe la parte comentada en otra frase.}
Este segundo caso de uso representa una exigencia mayor al tener que compartir la deformación producida con otra librería manteniendo las tasas de refresco interactivas. La validación de apariencia y de contenido realizada para el simulador de radiología diagnóstica permite asegurar que el algoritmo proporciona deformaciones útiles para el entrenamiento de las proyecciones radiológicas.
Esto induce a pensar que se pueden utilizar enfoques geométricos para el entrenamiento de procedimientos médicos.

Considerando esto último, se puede afirmar positivamente que se ha validado la hipótesis de partida.




\section{Contribuciones}
\subsection{Contribuciones científicas}
\label{conclu:cientifica}

\textbf{Congreso nacional } \emph{``An Interactive Algorithm for Virtual Patient Positioning"} \cite{ceig.20151197} presentando en Congreso Español de Informática Gráfica en 2015. 

\textbf{Artículo científico }\emph{``Real-time animation of human characters' anatomy"}\cite{SUJAR2018268} presentando en la revista Computer \& Graphics en 2018. 

\todo{Paper EG y el paper pendiente de publicacion.}

La primera contribución fue presentar una versión inicial del algoritmo en un congreso nacional. El objetivo era la adaptación del cauce clásico de animación esqueletal a modelos volumétricos. En esta publicación, la técnica de \emph{skinning} utilizada es \ac{DQS} que, aunque es una solución recurrente al método clásico \ac{LBS}, generaba incrementos de volumen que no eran útiles para utilizarlos en entrenamiento médico. De esta manera se introdujo un proceso de optimización no interactivo a cambio de mejorar la calidad del resultado. Mientras se trabajaba en las distintas integraciones del algoritmo en los casos de uso presentado, se buscó alternativas a la técnica \ac{DQS}.

Así, como segunda contribución, se presentó un artículo en una revista impactada, en el que se presentaba el algoritmo propuesto. Como se ha descrito en los resultados del capítulo (sec. \ref{posing:result}), la solución propuesta al utilizar \ac{COR} podría reemplazar la fase de optimización con la consecución de que la herramienta permita deformaciones útiles para el entrenamiento médico interactivamente.






\subsection{Contribuciones técnicas}
\label{conclu:tecnica}
\todo{Aqui no hables de paper. Habla de herramientas desarrolladas, librerias: Libreria de posing, integradada con GMRVGL (libreria de render desarrollada en el grupo), herramienta de posing. Integracion de la herramienta de creacion de pacientes virtuales. Integración de RASim. Adapatacion del prototipo a Flockof birds (has reescrito los drivers). Herramienta de radiologia diagnostica.}

\textbf{ Congreso Internacional }\emph{``A Virtual Physiological Human Model for Regional Anaesthesia"}\cite{VHZKLBSGSD16} presentado en la conferencia Virtual Physiological Human (VPH) en 2016.

La primera contribución técnica es la inclusión del algoritmo propuesto dentro del proyecto \ac{RASimAs}. La idea principal era crear una herramienta que permita crear una base de datos que contenga multitud de variaciones anatómicas.  Se publicó inicialmente esta publicación en el congreso \emph{ Virtual Physiological Human (VPH)} como introducción a las demás publicaciones relacionadas con el proyecto \ac{RASimAs}. Lamentablemente, no se ha podido seguir en esta línea hasta que el simulador haya pasado su validación clínica.

\textbf{ Congreso Internacional }\emph{``gVirtualXRay: Virtual X-Ray Imaging Library on GPU"}\cite{sujar:hal} presentando en el congreso Computer Graphics and Visual Computing en 2017.

Esta contribución surge como colaboración con \emph{Franck P. Vidal} \cite{gVirtualXRay}, el cual dirige un proyecto de simulación de rayos X \emph{gVirtualXRay}. Esta colaboración dio lugar al segundo caso de uso de esta tesis. En este artículo, se presentó las posibilidades de la librería de \emph{gVirtualXRay} para ser utilizada como simulador de radiología que será finalmente completada con la publicación del artículo en progreso llamado: \emph{``Interactive learning environment for diagnostic radiographywith real-time X-ray simulation and patient positioning"} con intención de ser publicado en la revista impactada Computerized Medical Imaging and Graphics en un futuro cercano. 


\todo{Mete como contribucion cientifica el ultimo paper, ponlo como apendice y di que está pendiente de publicar. }


\section{Limitaciones y líneas futuras}
\label{conclu:future}

La principal limitación que presenta el algoritmo de posicionamiento de pacientes virtuales es la utilización de un enfoque geométrico. Como tal, los resultados que proporcionan son heurísticos y no físicamente correctos. Esta característica impide que pueda emplearse para planificación quirúrgica o en aquellos lugares que necesiten un comportamiento fiel a la realidad. 
\new{Revisar:y, aunque el simulador no tiene en cuenta efectos como la gravedad,}

Aún así, los resultados obtenidos son lo suficientemente plausibles para proporcionar un método que consiga generar modelos anatómicos en el contexto de entrenamiento médico.

Se puede destacar que el diseño seguido del cauce de animación esqueletal permite actualizar los algoritmos utilizados en cada etapa independientemente. En el futuro pueden sustituirse etapas por nuevos métodos que permitan mejorar los resultados, incluso se podría incorporar nuevos métodos basados en física que permitan animación en tiempos interactivos.

\todo{Comenta la posibilida de cinematica inverasa a la herramienta de seleccion de poses. Di que es trivial. Pon una cita.}

Aunque el proyecto \ac{RASimAs} se completó de manera favorable, la falta de validación clínica es un requisito fundamental para la publicación de resultados. Se espera poder solucionar los defectos que presentan los dispositivos o incorporar otros diferentes con el fin de poder llevar el simulador a los entornos clínicos.
Además, este entorno se podría adaptar a otro tipo de procedimientos médicos. Las adaptaciones hechas con tecnología de impresoras 3D para crear los instrumentas médicos permiten adaptar el entorno de trabajo a otros tipos de instrumentos y procedimientos médicos.

En cuanto al simulador de diagnóstico radiológico, esta afectado por las limitaciones propias del algoritmo propuesto. Las deformaciones producidas son utilizadas como ejemplos de entrenamiento y no sirve para planificación médica. Aquellas mejoras realizadas en el método propuesto tendrán un beneficio automático en este simulador.

Por otra parte, se encuentran las limitaciones de la librería \emph{gVirtualXRay}. La configuración de la intensidad \acs{mAs} no está incorporada en el sistema, y es un parámetro importante para los radiólogos con el fin de mejorar la calidad de la imagen radiológica. Además, podría incorporarse la funcionalidad de producir radiografías con modelos volumétricos, representaciones que el algoritmo propuesto si puede manejar.

Otra funcionalidad extra que se podría incluir en el simulador, es la medición de la dosis recibida por el paciente. Existen actualmente simuladores de dosis y calculadoras \emph{online} \cite{xraydose}, que podrían ser incorporadas al simulador. De la misma manera, es posible incorporar otras técnicas de imagen médica que puedan ser simuladas. Se podría ampliar el abanico de tecnologías de diagnóstico por imagen (p. ej. \acs{IRM}, \acs{US} o \acs{TC}) que requieran posicionar al paciente de la misma manera que las proyecciones radiológicas, lo cual podría derivar en una nueva línea de investigación.


Por último hay que destacar que en ninguno de los dos simuladores, tanto \ac{RASim} como el simulador de radiología diagnóstica, tienen en cuenta los procedimientos protocolarios. La preparación de la sala o instrumental médico, aspectos legales o burocráticos, uso de lenguaje apropiado o aspectos de la técnica aséptica fundamentales en una situación real, son aspectos importantes del procedimiento que no son cubiertos por los simuladores. 







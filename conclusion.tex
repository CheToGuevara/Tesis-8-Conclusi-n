\chapter{Conclusiones}
\label{cap:conclu}

\section{Algoritmo de posicionamiento de pacientes virtuales}
\label{conclu:posing}

En la actualidad, la \ac{RV} está demostrando su potencial como herramienta de entrenamiento en el campo de la medicina. %En este contexto, el algoritmo propuesto ofrece la posibilidad de adaptar la postura de cualquier paciente virtual a la posición requerida por el procedimiento médico que se va a simular.
En esta tesis se ha presentado un cauce de animación esqueletal semiautomático que permite transformar cualquier paciente virtual con estructuras internas a la posición requerida en el entrenamiento de un determinado procedimiento médico. Las primeras etapas de este cauce se realizan en un preproceso automático que generan la información necesaria para que, a continuación, el usuario pueda seleccionar la postura del paciente virtual interactivamente.

Esta técnica sigue un enfoque geométrico y, por tanto, proporciona una solución heurística que no refleja \del{completamente} el comportamiento físico de los tejidos. Aunque recientemente las simulaciones músculo-esqueletales han avanzado \new{mucho} en el mundo de los gráficos por computador, estas técnicas requieren de una descripción, \new{tanto mecánica como geometríca}, detalla y completa de los tejidos a animar, los cuales en ocasiones no se encuentran disponible. 
%quitar salto de párrafo
\del{Por tanto,} Una de las ventajas del algoritmo propuesto es la posibilidad de transformar un paciente virtual junto con sus tejidos internos aunque estos se encuentren incompletos o falte su descripción mecánica. El algoritmo se ha diseñado para ser robusto y poder adaptar la posición de cualquier modelo anatómico. Además, se permite utilizar tanto mallas superficiales como modelos volumétricos. Únicamente, se considera necesario el tejido óseo y la piel del modelo anatómico debido a que la animación se dirige por los huesos que contenga el paciente virtual.
%quitar salto de párrafo
Otra ventaja que presenta el algoritmo propuesto es \new{su eficiencia computacional que permiten utilizarlo en aplicaciones interactivas}. El usuario podrá seleccionar la pose del paciente virtual interactivamente, ya que se ha diseñado \new{el cauce de tal forma que las etapas no que no pueden ejecutarse de en tiempo real se realicen de forma automática en preproceso.} \del{permite una automatización de las tareas más complejas}.
%Yo quitaría esta última frase. No se muy bien que aporta. 
\del{Cabe destacar las innovaciones incluidas en cada etapa del cauce de animación esqueletal.}

\new{La etapas del cauce propuesto son suficientemente flexibles como para poder adaptarse una gran diversidad de problemas, siendo la fase de \emph{rigging} la que más limitaciónes impone.
La fase de \emph{rigging} es una solución especifica diseñada para integrase en el generador de pacientes virtuales de RASim. Aun así } este método puede ser fácilmente \new{extendido siempre y cuando se pueda establecer una correspondencia directa entre el tejido oseo y el esqueleto virtual.}
%ampliado, aunque el único requerimiento es haya una una correspondencia directa entre el hueso virtual y los tejidos óseos  presentes en el modelo anatómico. Si no fuera así, se podría generar deformaciones no realistas.

En cuanto a la fase de pesado, se ha adaptado la  propuesta de \emph{Baran y Popovi\'{c}}. En \cite{Baran:2007} se proponía utilizar la ecuación de calor para calcular los pesos automáticamente. En el cauce propuesto, esta ecuación ha sido adaptada para ser utilizada en mallas volumétricas en comparación con las mallas superficiales usadas originalmente. \new{Además, nuestra técnica utiliza la geometría del tejido oseo en lugar de la geometría del esqueleto virtual. }

%Hay que señalar los resultados obtenidos en la fase de \emph{skinning}. 
\new{La etapa de \emph{skinning}} consigue tasas de refresco interactivas altas, incluso en modelos con alto grado de detalle. \new{Frente a los métodos basados en modelos físicos más eficientes, el algoritmo propuesto ofrece un rendimiento un orden de magnitud superior.}
%Esta complejidad es en ocasiones de un orden superior en comparación con los métodos de animación basados en físicas. 
%Por una parte, modelos precisos raramente consiguen tiempos interactivos, y por otra parte, recientemente aparecidos modelos basados en dinámicas de puntos usan modelos mucho más complejos para ser interactivos  \cite{abu2015position}.
%Quitar salto de parrafo
\todo{Comenta algo de la adapatación de CoR, porque se escogio esta técnica y comenta que puede utilizarse cualquier técnica de skinning basada en pesos.}


Siendo conscientes de las limitaciones de los métodos de \emph{skinning}, se incorporó una fase de optimización que pretendía mejorar los resultados obtenidos. Esta etapa consiste en utilizar un modelo matemático basado en mecánica de medios continuos, \new{con el único objetivo de garantizar la conservación del volumen}.
%Sin poder utilizar propiedades mecánicas  del modelo, el objetivo era garantizar la conservación del volumen en la malla volumétrica. 
%Esta etapa podría ser modificada para tomar en cuenta las propiedades mecánicas de los tejidos en el momento de generar la malla volumétrica y así poder mejorar el realismo del método.

Es importante también tener en cuenta que una solución heurística no produce resultados físicamente correctos. Es por esta razón que esta técnica no es adecuada para planificación quirúrgica. Sin embargo, los entrenadores médicos no necesitan específicamente modelos de pacientes reales sino una muestra variada de modelos anatómicos que permitan entrenar casos plausibles. Estos modelos se podrán utilizar en herramientas educativas y de aprendizaje para formar a estudiantes de una manera innovadora que no proporcionan los materiales clásicos como los libros.
%Sin salto de parrafo
Para demostrar las distintas capacidades del algoritmo presentado, se han presentado dos casos de uso que permiten hacerse una idea de las capacidades del método para incorporarse en soluciones de \ac{RV}. 
%
Por una parte, se ha incorporado el algoritmo en una \emph{suite} de aplicaciones que permite \new{crear una base de datos de pacientes virtuales.} 
%incorporados en el simulador \ac{RASim}. 
\ac{ITGVPH} crea \new{modelos que representen distintas  variaciones anatómicas regsitrando un modelo de paciente virtual e imágenes médicas promedio.} 
%procedentes del registro entre un modelo comercial e imágenes médicas. 
El algoritmo permite 
%modificar % no se modifica "a", se adapta "a"
\new{adaptar} la postura de este modelo a la requerida por el simulador.
%
Por otra parte, el algoritmo se ha incorporado en simulador de radiología diagnóstica. \new{Esta plataforma requiere que el usuario modifique la pose del modelo virtual de forma interactiva como parte del procedimiento simulado.} \del{que necesita deformaciones en tiempo real y permitan al usuario interaccionar con los modelos virtuales.}


Por último, aunque los casos de uso están orientados en el campo de la medicina, esta técnica \new{no esta limitada a esta área y }podría ser empleada para animar modelos anatómicos con estructuras internas en otros campos como los videojuegos, la industria del cine, etc.
 

\section{RASimAs}


\subsection{Herramienta ITGVPH}
\label{conclu:herramienta}
El origen de la creación de la herramienta  \ac{ITGVPH} es la generación de una base de datos de \ac{VPH} para el proyecto \ac{RASimAs}. Esta herramienta proporcionaría la ventaja de permitir a los estudiantes practicar con multitud de escenarios de forma segura. La creación una base de datos con pacientes virtuales puede ser clave en la mejora de los simuladores de \ac{RV} para el campo de la medicina. 

%Esta herramienta está compuesta por tres módulos desarrollados por separado. Uno de ellos es el algoritmo de posicionamiento de pacientes virtuales que se ha presentado en esta tesis. Estos módulos han sido creados de manera independiente.
Esta herramienta es capaz de generar nuevos \ac{VPH} a partir de modelos  anatómicos \new{viruales} existentes y de imágenes médicas que proporcionen variabilidad anatómica. Estos pacientes virtuales se pueden transformar para que tengan la postura necesaria para el entrenamiento del procedimiento médico. %Por último, se generarán tanto los modelos superficiales como volumétricos con los parámetros físicos que se quieran simular. Estos modelos describen al \ac{VPH} anatómicamente y mecánicamente para que puedan ser utilizados en el proyecto europeo. Además, esta herramienta permite la ejecución en serie o de manera independiente los módulos. Debido a que en ocasiones pueda ser posible utilizar ciertas etapas sin necesidad de ejecutar el cauce completamente.

Esta herramienta es semiautomática debido a que se requiere limitar la interacción del usuario con un perfil no técnico a la selección de poses. Según los parámetros de entrada, la herramienta ejecuta los módulos secuencialmente, requiriendo la interacción del usuario para seleccionar la postura del paciente virtual. El supervisor clínico puede generar tantas posturas como sea necesario a través de una \ac{IU} que muestra la escena 3D donde se puede observar el paciente virtual. Una vez seleccionado, la herramienta sigue con las tareas para generar los modelos que requiere el simulador \ac{RASim}.

\todo{indica que has creado un la aplicación de selección de poses que usa tu algoritmo. Indica que has trabajado en en la integración del entorno.}
%En esta, se puede configurar las etapas de manera independiente. El usuario es capaz de seleccionar el modelo anatómico de referencia (p. ej Zygote), las imágenes médicas en formato \ac{DICOM} (p. ej imágenes de \ac{IRM}), la postura necesaria para el procedimiento y, por último, especificar parámetros biomecánicos. 
%Las posturas de los \ac{VPH} son definidas por un En esta interfaz, el usuario podrá interaccionar con las articulaciones disponibles para generar unas posiciones físicamente correctas.

\subsection{RASim}
\label{conclu:rasim}
\todo{3 Apunta en las conclusiones una discusión sobre los problemas de RAsimas que se alcanzo que no y porque.}
El simulador \ac{RASim} proporciona un entorno de entrenamiento muy completo y realista para el entrenamiento del procedimiento de \ac{RA} guiado por \ac{US}.

Este simulador cuenta con una mesa de trabajo donde el usuario puede realizar el procedimiento con instrumentos \new{que imitan un entrono real}
%parecidos a los reales. 
En los monitores podrá supervisar la interacción de estos y el paciente, mientras puede observar la imagen de \ac{US}. Así, el médico podrá guiar la aguja hasta el nervio a través de lo que ve en la pantalla. El \ac{Courseware} se ha diseñado para ofrecer una plataforma de entrenamiento y autoevaluación. Permite al usuario interaccionar con el sistema de manera más efectiva y mejorar su aprendizaje al dirigirle en las tareas que tiene que realizar. Este módulo también monitoriza su rendimiento con el objetivo de ofrecer retroalimentación. Las métricas registradas se utilizan para su uso como evaluación formativa y sumativa. 

\ac{RASim} permite al estudiante mejorar sus habilidades cognitivas y no cognitivas. La práctica de del procedimiento permitirá al usuario enfrentarse con mayor seguridad en los primeros pasos de su carrera clínica. 



\todo{
Aaron tienes que ser mas estructurado. Te pongo las ideas que tienes que contar y tu las das formas:
1. Dos objetivos importantes:
1.1 Proporcionar retroalimentacion haptica. Es importante porque ayuda a localizar la aguja
1.2 Mantener el precio del simulador bajo. 
2. Se escogió el dispositivo phantom (deberías haberlo contarlo en Rasimas cuando) proporcionar un dispositivo haptico de bajo precio. Cumple los dos objetivos
3. Se encontraron problemas en los nuevos dispositovos (pon un anexo con la carta de  Geomagic indicando el error). Yo daria detalles explicando los problemas de precisión.
4. Esto problema impidieron la validación del prototipo y por lo tanto de la hipótesis. 
5. La URJC y el doctorando apataron apataron el prototipo para utilizar un dispositivo flock of birds. 
Este dispositivo no proporciona retroalimentacion haptica pero es muy precsiso. 
6. Los socios medicos validaron este prototipo a pesar de no tener retroalimentación.
7. Lamentablemente se consumieron los plazos y puedo concluierse el resto de evaluaciones que estaban probramadas quedandonos soloamanete con el visto bueno de los socioes medicos. }
Uno de los requisitos del proyecto \ac{RASimAs} era poder simular la sensación háptica de la aguja. Esta funcionalidad era muy esperada por el comité clínico, porque esto implicaría ayudar a los estudiantes a ser más eficientes en el procedimiento. Lamentablemente, este módulo no fue posible incorporarlo al simulador debido a diversos problemas, el comité médico no permitió su utilización por parte de los estudiantes y por tanto, no se pudo realizar una validación clínica del sistema. 
%
Aunque se propuso una solución utilizando \ac{tracker}s en lugar de los dispositivos hápticos, el proyecto europeo finalizó sin poder realizar la validación clínica, con la cual permitiría asegurar que el simulador \ac{RASim} proporcionaría un entorno de entrenamiento completo para estudiantes o profesionales de la anestesia. 

%a través de los dispositivos utilizados en el simulador permiten hacerse una idea de que se van a encontrar en un procedimiento real. 





\section{Simulador de radiología diagnóstica }
\label{conclu:xray}

Se ha presentado un simulador 
%de
\ac{RV} que permite enseñar y aprender el procedimiento de radiología diagnóstica. Los técnicos de radiología deben conocer como colocar al paciente y configurar la máquina de rayos X para evitar repeticiones del procedimiento y \new{reducir la dosis de radiación que recibe el paciente.}
%innecesaria para el paciente. 
De esta manera, el simulador proporciona un entorno seguro donde el procedimiento puede ser ensayado sin ningún riesgo para el médico o pacientes. 

Esta aplicación se ha desarrollado para servir como herramienta complementaria para el aprendizaje y la docencia del procedimiento de realizar proyecciones radiológicas. En este simulador, el usuario puede practicar las proyecciones radiológicas deformando al paciente virtual interactivamente y sus tejidos internos mientras ve una imagen radiográfica en tiempo real. 

Es importante destacar que el simulador propuesto no sustituye el método tradicional de enseñanza, sino que mejora el proceso de aprendizaje al proporcionar un entorno interactivo. Frente a los archivos educativos y los libros que proporcionan los conocimientos teóricos, los simuladores permiten practicar el procedimiento interactivamente para conseguir mejorar las habilidades no cognitivas. En este simulador, el usuario puede replicar cualquier proyección que pueda leer de un libro, pero a la vez, se permite interaccionar con el mismo paciente virtual, variar configuraciones de la maquina de rayos X e, incluso, generar enfermedades en el modelo anatómico.




%Este simulador es una unión entre el algoritmo de posicionamiento de pacientes virtuales y la librería de simulación de rayos X. Este algoritmo es susceptible de ser modificado para mejorar las limitaciones que vienen acompañadas del uso de una técnica geométrica.  








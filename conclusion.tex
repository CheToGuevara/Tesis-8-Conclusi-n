\chapter{Conclusiones}
\label{cap:conclu}

\section{Algoritmo de posicionamiento de pacientes virtuales}
\label{conclu:posing}
En la actualidad, la \ac{RV} está demostrando su potencial como herramienta de entrenamiento en el campo de la medicina.  Una de las principales ventajas es permitir a los estudiantes practicar en multitud de escenarios de forma segura. La creación una base de datos con pacientes virtuales puede ser clave en la mejora de los simuladores de \ac{RV} para el campo de la medicina. En este contexto, el algoritmo propuesto ofrece la posibilidad de adaptar la postura de un modelo virtual a la requerida por un procedimiento médico. Aunque recientemente las simulaciones músculo-esqueletales han avanzado en el mundo de los gráficos por computador, estas técnicas requieren de una descripción detalla y completa de los tejidos a animar, los cuales raramente se encuentran disponible. En esta tesis se ha presentado un cauce de animación que permite transformar cualquier modelo anatómico con estructuras internas a la posición final. Esto es debido a que a técnica propuesta sigue un enfoque geométrico y, por tanto, proporciona una solución heurística que no refleja completamente una solución físicamente correcta.

Esta técnica representa una automatización del cauce clásico de una animación esqueletal y permite la animación de los tejidos internos. Esto significa que este algoritmo es robusto ante la falta de descripciones mecánicas o modelos anatómicos incompletos. Además, el método propuesto da la flexibilidad de usar mallas superficiales o modelos volumétricos. Únicamente, se considera el tejido oseo imprescindible ya que la animación estará dirigida por los huesos que contenga el modelo virtual.

En particular, la fase de \emph{skinning} del cauce es muy rápida en comparación con los modelos de animación basados en físicas. Por una parte, modelos precisos raramente consiguen tiempos interactivos, y por otra parte, recientemente aparecidos modelos basados en dinámicas de puntos usan modelos mucho más complejos para ser interactivos  \cite{abu2015position}.

Por otra parte, la fase de \emph{rigging} es una solución especifica integrada en algoritmo propuesto. Sería posible incluir otros métodos automáticos como \cite{Baran2007} o seguir permitiendo al usuario proceder manualmente. El único requerimiento es la correspondencia entre el hueso virtual y los tejidos óseos presentes en el modelo ya que si no fuera así, podría generar deformaciones no realistas.

Siendo conscientes de las limitaciones del método geométrico de la fase de \emph{skinning}, se incorporó una fase de optimización que pretendía mejorar los resultados obtenidos. Esta etapa consiste en utilizar un modelo matemático basado en mecánica de medios continuos. Sin poder utilizar propiedades mecánicas  del modelo, el objetivo era garantizar la conservación del volumen en la malla volumétrica. Esta etapa puede ser modificada para tomar en cuenta las propiedades mecánicas de los tejidos en el momento de generar la malla volumétrica y así poder mejorar el realismo del método.


Es importante tener en cuenta que una solución heurística no produce resultados físicamente correctos. Es por esta razón que esta técnica no es adecuada para planificación quirúrgica. Sin embargo, los entrenadores médicos no necesitan específicamente modelos de pacientes reales sino una muestra variada de modelos anatómicos que permitan entrenar casos plausibles. Herramientas educativas y de aprendizaje pueden utilizar estos modelos para formar a estudiantes de una manera innovadora que no proporcionan los materiales clásicos como los libros.





Para demostrar las distintas capacidades del algoritmo presentado, se han presentado dos casos de uso que permiten hacerse una idea de las capacidades del método para incorporarse en soluciones de \ac{RV}. Por una parte, se ha presentado un simulador médico que necesita generar variabilidad anatómica para ser usados en el propio sistema. Estos modelos son generados a partir de otros modelos virtuales e imágenes médicas. Por otra parte, el algoritmo puede ser incorporado en simuladores que necesiten deformaciones en tiempo real y permitan al usuario interaccionar con los modelos virtuales.
Es importante remarcar que, aunque los ejemplos de casos de uso del método han sido orientados al campo de la medicina, esta técnica puede ser empleada para animar modelos anatómicos con estructuras internas en otros campos como los videojuegos, la industria del cine, etc...
 

\section{RASimAs}


\subsection{Herramienta \emph{offline}}
\label{conclu:herramienta}
El origen de la creación de la herramienta \emph{offline} \ac{TPTVPH} es la generación de una base de datos de \ac{VPH} para el proyecto \ac{RASimAs}. Esta herramienta está compuesta por tres módulos desarrollados por separado. Uno de ellos es el algoritmo de posicionamiento de pacientes virtuales que se ha presentado en esta tesis. Estos módulos han sido creados de manera independiente con el objetivo de simplificar la consecución de una tarea tan compleja y su desarrollo en paralelo. Esta herramienta es capaz de generar nuevos \ac{VPH} a partir de modelos anatómicos existentes y la utilización de imágenes médicas que proporcionen variabilidad anatómica. Si es necesario, estos nuevos modelos son reposicionados a la postura necesaria para el procedimiento a simular. Por último, se generarán tanto los modelos superficiales como volumétricos con los parámetros físicos que se quieran simular. Estos modelos describen al \ac{VPH} anatómicamente y mecánicamente para que puedan ser utilizados en el proyecto europeo. Además, esta herramienta permite la ejecución en serie o de manera independiente los módulos. Debido a que en ocasiones pueda ser posible utilizar ciertas etapas sin necesidad de ejecutar el cauce completamente.
Se ha diseñado una interfaz de usuario que permite una fácil interacción con la herramienta. En esta, se puede configurar las etapas de manera independiente. El usuario es capaz de seleccionar el modelo anatómico de referencia (p. ej Zygote), las imágenes médicas en formato \ac{DICOM} (p. ej imágenes de \ac{IRM}), la postura necesaria para el procedimiento y, por último, especificar parámetros biomecánicos. 
Las posturas de los \ac{VPH} son definidas por un supervisor clínico que puede generar tantas posturas como sea necesario a través de una interfaz de usuario que muestra la escena 3D donde se puede observar el modelo anatómico cargado. En esta interfaz, el usuario podrá interaccionar con las articulaciones disponibles para generar unas posiciones físicamente correctas.

\subsection{RASim}
\label{conclu:rasim}
\todo{3 Apunta en las conclusiones una discusión sobre los problemas de RAsimas que se alcanzo que no y porque.}
El simulador \ac{RASim} proporciona un entorno de entrenamiento muy completo y realista para el entrenamiento del procedimiento de \ac{RA}.  Este simulador cuenta con una mesa de trabajo que le hace sentir al usuario inmerso en el procedimiento y por tanto mejorar sus habilidades cognitivas y no cognitivas. La disposición de unos instrumentos lo más parecido a la realidad permiten que el usuario se 

Uno de los puntos fuertes del simulador es la generación de imágenes de \ac{US} realistas y que permitan al usuario practicar su interpretación. La imagen de \ac{US} es importante en el procedimiento de \ac{RA} porque el médico guía la aguja hasta el nervio a través de lo que ve en la pantalla. La práctica de estos movimientos permitirán al usuario enfrentarse con mayor seguridad en los primeros pasos de su carrera clínica. 

Por otra parte, la sensación háptica que experimenta el usuario a través de los dispositivos utilizados en el simulador permiten hacerse una idea de que se van a encontrar en un procedimiento real. Esta sensación es una funcionalidad muy esperada por el comité clínico, porque esto implicaría ayudar a los estudiantes a ser más eficientes en el procedimiento.


Por último, el \ac{Courseware} permite al usuario interaccionar con el sistema de manera más efectiva y mejorar su autoaprendizaje al dirigirle en las tareas que tiene que realizar. Este módulo también monitoriza su rendimiento con el objetivo de retro alimentar el aprendizaje, y siendo posible además que las métricas registradas puedan usarse como método de evaluación del estudiante.

Después de una validación clínica, \ac{RASim} proporcionaría un entorno de entrenamiento completo para estudiantes o profesionales de la anestesia. 




\section{Simulador de radiología diagnóstica }
\label{conclu:xray}
Esta aplicación ha sido desarrollada para servir como herramienta complementaria para el aprendizaje y la docencia del procedimiento de realizar proyecciones radiológicas. Este simulador permite simular protocolos o métodos para realizar el diagnóstico por imagen, permitiendo al usuario probar cualquier elemento del entorno simulado. Esta herramienta como material complementario puede ser útil para mejorar el aprendizaje de los futuros radiólogos.

Este simulador es una unión entre el algoritmo de posicionamiento de pacientes virtuales y la librería de simulación de rayos X. Este algoritmo es susceptible de ser modificado para mejorar las limitaciones que vienen acompañadas del uso de una técnica geométrica.  

De la misma manera, es posible incorporar otras técnicas de imagen médica que puedan ser simuladas. Ampliar el abanico de tecnologías de diagnóstico por imagen (p. ej. \ac{IRM} o \ac{TC}) que requieran posicionar al paciente de la misma manera que las proyecciones radiológicas. 





